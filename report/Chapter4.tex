\chapter{Results}

In the following, the results of the speed ups attained by optimizing using CUDA will be shown. However, before this can be done, some preliminaries need to be listed. Firstly, all hardware testing was done on a single PC with an Intel Core 2 Duo T8100 Processor with a clock speed of 2.1 GHz and 4 GB of RAM. The graphics hardware used was the NVIDIA GeForce 8600M GT, with CUDA 2.1 software installed. It should be noted that at the time of writing, CUDA 2.2 was available although in beta form and was not chosen due to potential instabilities. Timing code used was from the \texttt{cutil} library provided in the CUDA toolkit.

Although 8600M GT is adequate for CUDA development, it has rather limited performance in comparison to other graphics chips. This implies that the performance speed ups below could potentially be up to a further order of magnitude faster on latest hardware. For example, although the shader processing rate of 8600M GT is quoted as 91.2 Gigaflops the recently released GeForce GTX 295 boasts an impressive 1788.48 Gigaflops potentially allowing for another order of magnitude speed up from the 8600M GT hardware. This is mainly due to the increased number of on chip multiprocessors, however to lesser extent is also due to the device being of higher \textit{compute capability}: there are fewer limitations (such as support for double precision arithmetic) and relaxed requirements for coalescing memory transfers. Due to the transparent runtime scalability of CUDA kernels, very few adjustments would need to be made to tailor code for new hardware. Nonetheless, the most obvious adjustment would be to increase the size of the thread blocks for increased occupancy.

\section{Speed Tests and Analysis}\label{results}

\subsection{2D Segmentations}


In Figure \ref{fig:liver} the example of a liver segmentation is shown. The liver data is of good contrast and dimension $256\times 256$ (which is a multiple of 16 implying no memory padding is required in CUDA). The liver has been entirely segmented, with the initial mask as the input. The time taken for 5000 iterations in MATLAB, C, CUDA (Unoptimized) and CUDA (Optimized) are shown in Table \ref{livertime}.

 
\begin{figure}[h]
  \begin{center}
    \subfigure[Feature Image Input - Liver]{\label{fig:liveroriginal}\includegraphics[scale=0.5]{images/liver.PNG}}
    \subfigure[Initial Mask Input]{\label{fig:livermask}\includegraphics[scale=0.5]{images/livermask.PNG}}
    \subfigure[Output of Segmentation]{\label{fig:liverseg}\includegraphics[scale=0.5]{images/liverseg.PNG}}
  \end{center}
  \caption{2D Liver Segmentation with parameters $T = 180, \epsilon = 45, \alpha = 0.003$}
  \label{fig:liver}
\end{figure}

\begin{table}[h]
\centering
\begin{tabular}{ | l | c | r | }
	\hline
	Algorithm Version  & Time (s)\\ \hline
  MATLAB 						 & 425.95 \\
  C 								 & 55.44 \\
  CUDA (Unoptimized) & 8.38 \\
  CUDA (Optimized)   & 1.73  \\
  \hline
\end{tabular}\caption{Comparison of runtime for different algorithm versions - 2D liver segmentation}
\label{livertime}
\end{table}

The runtime speed up attained from sequential code in C to CUDA optimized code is approximately $32 \times$. The block size used for 2D CUDA compute was $32 \times 8$.


\begin{figure}[h]
  \begin{center}
    \subfigure[Feature Image Input - brain]{\label{fig:brainoriginal}\includegraphics[scale=0.25]{images/brain.PNG}}
    \subfigure[Initial Mask Input]{\label{fig:brainmask}\includegraphics[scale=0.25]{images/brainmask.PNG}}
    \subfigure[Output of Segmentation]{\label{fig:brainseg}\includegraphics[scale=0.25]{images/brainseg.PNG}}
  \end{center}
  \caption{2D Brain Segmentation with parameters $T = 45, \epsilon = 30, \alpha = 0.003$}
  \label{fig:brain}
\end{figure}

In Figure \ref{fig:brain} we can see the brain in sagittal view. This image is of relatively poor contrast and has dimensions $512 \times 512$. This makes the image both a computationally demanding segmentation (as it has relatively large dimensions) and challenging in terms of accuracy.
The segmentation inputs and output can be seen in Figure \ref{fig:brain}. It can be seen that the sequential algorithm has performed reasonably in segmenting the white and gray matter and some of the brain stem. Considerable weighting had to be given to curvature in order to prevent leaks due to the poor contrast, resulting in a very rounded segmentation. 

\begin{table}[h]
\centering
\begin{tabular}{ | l | c | r | }
	\hline
	Algorithm Version  & Time (s)\\ \hline
  MATLAB 						 & 2737.84 \\
  C 								 & 322.81 \\
  CUDA (Unoptimized) & 44.68 \\
  CUDA (Optimized)   & 6.99  \\
  \hline
\end{tabular}\caption{Comparison of runtime for different algorithm versions - 2D brain segmentation}
\label{braintime}
\end{table}

The performance speedup attained on this larger image is therefore $46 \times$, which is greater than the speed up attained for the smaller $256 \times 256$ image. This motivates exploration into the effect of different image sizes on CUDA speed up, which is discussed in Section \ref{dimensions2d}. 

\subsubsection{Effect of Noise}

Denoising filters already exist as part of the CUDA SDK (i.e. \texttt{imageDenoising}). Our algorithm does not feature any image pre-processing algorithms such as denoising or blurring so its performance on noisy images is expected to be poor. In order to test this, artificial Gaussian noise of 20\% and 40\% was added to the liver image as shown in Figure \ref{fig:livernoise}. 

\begin{figure}[h]
  \begin{center}
    \subfigure[Liver with 20\% Noise]{\label{fig:livernoise20}\includegraphics[scale=0.4]{images/livernoise20.PNG}}
    \subfigure[Segmentation with $T = 180, \epsilon = 55, \alpha = 0.003$]{\label{fig:livernoise20seg}\includegraphics[scale=0.4]{images/livernoise20seg.PNG}}\\
    \subfigure[Liver with 40\% Noise]{\label{fig:livernoise40}\includegraphics[scale=0.4]{images/livernoise40.PNG}}
    \subfigure[Segmentation with $T = 180, \epsilon = 75, \alpha = 0.003$]{\label{fig:livernoise40seg}\includegraphics[scale=0.4]{images/livernoise40seg.PNG}}
  \end{center}
  \caption{Segmentation of the liver with artificially added noise}
  \label{fig:livernoise}
\end{figure}

It can be seen that through manipulating the parameter $\epsilon$ segmentations are still approximately valid. The effect of noise on performance of the algorithm was negligible.

\subsubsection{Effect of Different Image Sizes}\label{dimensions2d}

It is found, as expected, that for all versions of the algorithm (CUDA and single threaded) compute time scales linearly with the number of elements. For square 2D images, this implies that increasing image size by a factor of two in each dimension increases compute time by a factor of four. Figure \ref{fig:dimensions2d} displays compute times to 5000 iterations across the different algorithm versions. Both the same input image and mask were used for all tests, again with a block dimension of $32 \times 8$.

\begin{figure}[p]
	\begin{flushleft}
		\subfigure{\label{fig:dimensions2d}\includegraphics[scale=0.5]{images/dimensions2d.pdf}}
		\subfigure{\label{elements2d}\includegraphics[scale=0.5]{images/elements2d.pdf}}
	\end{flushleft}
	\label{fig:speed}\caption{Effect of different image sizes (a) On compute time (b) On number of elements computed per second]}
\end{figure}

It can be seen that the difference in compute time between the parallel and sequential versions is greatest for the largest images. This is due to the number of elements computed per second being much greater for the parallel algorithms than for the sequential algorithms, across all image sizes. Interestingly, Figure \ref{elements2d} shows that the number of elements computed per second is approximately constant for both the sequential and unoptimized CUDA implementations, but not for the optimized shared memory CUDA algorithm. This is due to very low occupancy of the GPU at these small image sizes, resulting in reduced masking of the high latency between device memory and shared memory.




\subsection{3D Segmentations}

\begin{figure}[h]
  \begin{center}
    \subfigure{\includegraphics[scale=0.25]{images/3devolution0.PNG}}
    \subfigure{\includegraphics[scale=0.25]{images/3devolution1.PNG}}
    \subfigure{\includegraphics[scale=0.25]{images/3devolution2.PNG}}
    \subfigure{\includegraphics[scale=0.25]{images/3devolution3.PNG}}
  \end{center}
  \caption{Level Set Surface Evolution in 3D at 0, 50, 150 and 200 Iterations}
  \label{fig:evolution}
\end{figure}

Figure \ref{fig:evolution} illustrates a level set surface evolving in 3 dimensions. In order to visualize the level set evolving every certain number of iterations the CUDA SDK example \texttt{volumeRender} was modified. This version of the code with visualization of the level set evolution is approximately a factor of 2 slower. This volume rendering engine uses ray tracing which is not advised for segmentation validation, and therefore \textit{Paraview 3.4.0} (www.paraview.org) was used.

The 3D segmentation CUDA code uses a block size of $32 \times 4$. A block size of $32 \times 8$ causes the kernel invocation to fail due to the registers used per threads multiplied by the thread block size being greater $N$ (where for G80 NVIDIA hardware $N=8192$ 32-bit registers per multiprocessor. This limits the extent to which occupancy can be increased to mask latencies due to global memory loads. Section \ref{blocksizes} explores the effect of varying block sizes on performance.

\begin{figure}[p]
  \begin{center}
    \subfigure{\includegraphics[scale=0.55]{images/brain3d.PNG}}
    \subfigure{\includegraphics[scale=0.55]{images/brain3dseg3.PNG}}
    \subfigure{\includegraphics[scale=0.55]{images/brain3dseg1.PNG}}
    \subfigure{\includegraphics[scale=0.55]{images/brain3dseg2.PNG}}
  \end{center}
  \caption{Segmentation of a brain MRI dataset with parameters $T = 150, \epsilon = 50, \alpha = 0.03$ MRI data from \cite{brainweb}}
  \label{fig:brain3d}
\end{figure}
\begin{figure}[p]
  \begin{center}
    \subfigure[]{\includegraphics[scale=0.4]{images/heart3dseg.PNG}}
    \subfigure[]{\includegraphics[scale=0.4]{images/heart3dseg1.PNG}}
  \end{center}
  \caption{Segmentation of the right and left ventricles from a heart MRI dataset with parameters $T = 180, \epsilon = 60, \alpha = 0.02$ (a) Input data slice (b) Segmented heart clipped through $z$ plane}
  \label{fig:heart3d}
\end{figure}

The results from segmenting the cerebral hemispheres, cerebellum and brain stem can be seen in Figure \ref{fig:brain3d}. The high quality of this segmentation is due in part to the excellent \textit{BrainWeb} MRI data used (data is available from \cite{brainweb}). This data is of size $181 \times 217 \times 181$ and is therefore a good test of the algorithm for reasonably large volumes. As the CFL condition had not been implemented in the 3D level set solver, it was found to converge at 1000 iterations and that $DT$ values greater than 0.1 resulted in instability.

The times taken to segment are shown in Table \ref{brain3dtime}. MATLAB performance is not shown as out of memory errors were encountered when loading such large arrays (and even if these had not been encountered, the segmentation would have taken an infeasible amount of time).

\begin{table}[h]
\centering
\begin{tabular}{ | l | c | r | }
	\hline
	Algorithm Version  & Time (s)\\ \hline
  MATLAB 						 & N/A \\
  C 								 & 4697.5 \\
  CUDA (Unoptimized) & 392.8 \\
  CUDA (Optimized)   & 141.2  \\
  \hline
\end{tabular}\caption{Comparison of runtime for different algorithm versions - 3D Brain segmentation of BrainWeb data \cite{brainweb}}
\label{brain3dtime}
\end{table}

An impressive speed up of $33 \times$ is observed. To further demonstrate the power of this algorithm testing was briefly done on a more mid-range 8800 GTX card, observing a speed up of $117 \times$ compared to the sequential algorithm.

In Figure \ref{fig:heart3d} segmentation of both the right and left ventricles can be seen. This segmentation data only had 17 $z$ plane slices (total resolution $256 \times 160 \times 17$).


\subsubsection{Effect of Different Volume Sizes}\label{dimensions3d}


Figure \ref{fig:3dvolumesizes} shows the effect of multiple volume sizes on the compute time to 1000 iterations on the optimized CUDA algorithm. Tests were not run on unoptimized CUDA code as this is not of particular interest.


%\begin{table}[h]
%\centering
%\begin{tabular}{ | c | c | c | }
	%\hline
	%Volume Dimensions  & Elements / Second & Time (s)\\ \hline
  %$64 \times 64 \times 64  $	 & 54050.3			& 4.9 \\
  %$128 \times 128 \times 128  $&	53676.8		& 39.1 \\
  %$128 \times 256 \times 128  $& 53092.5   & 79.0 \\
  %$128 \times 128 \times 256  $&	53485.1   & 78.4  \\
  %$256 \times 256 \times 256  $&	52925.0   & 317.1  \\
  %\hline
%\end{tabular}\caption{Comparison of runtime for different volume sizes}
%\label{table:dimensions3d}
%\end{table}

\begin{figure}
	\centering
		\includegraphics[scale=0.6]{images/3dvolumesizes.pdf}
	\caption{Number of elements computed per second for different volume sizes}
	\label{fig:3dvolumesizes}
\end{figure}


It can be seen from this figure that the speed up at low volume sizes is much smaller than the speed up at larger volume sizes. The sequential algorithm performs almost half as slowly as itself for volume sizes larger than $64^3$. This is most likely due to the fast on board CPU cache being used only for volume sizes smaller than this, for volume sizes larger it cannot fit on the CPU cache and so is stored on the slower DRAM. Conversely, the CUDA code performs relatively poorly for small volume sizes and much more quickly for larger ones. This is essentially due to low numbers of processors being used for such small images and many more being used for larger images. Therefore the speedup line essentially shows that the algorithm follows Amdahl's and Gustafson's laws of parallel computation.

Volume sizes much larger than $256^3$ could not be tested as the maximum amount of global memory available for the 8600M GT is 256 MB. For example, a $320^3$ volume would take up $320^3 \times \texttt{sizeof(float)}$ and there are three of these arrays (for the feature image, previous level set iteration and current level set iteration) which would take up 375 MB of graphics memory. It is however expected, for these even larger volumes, that the speedup of the algorithm will remain approximately plateaued. 

\subsubsection{Effect of Different CUDA Block/Grid Sizes}\label{blocksizes}

Finding the optimum block size for CUDA code is one of the most important ways to optimize performance. Block sizes should not be a user parameter (other than for testing purposes most notably across different GPUs) as it assumed the developer would have chosen the optimum block size for maximum performance for a particular GPU. Figure \ref{table:blocksizes} shows the compute times to 1000 iterations for the CUDA optimized code with different block sizes, all other parameters were held constant. 

\begin{table}[h]
\centering
\begin{tabular}{ | c | c | c | }
	\hline
	$BX \times BY$  & Threads/Block & Time (s) \\ \hline
  $32 \times 4$	 &128			& 141.7 \\
  $16 \times 8$	 &128			& 141.9 \\
  $16 \times 12$ &192 & 108.4 \\
  $32 \times 6$	 &192   & 107.4  \\
  $48 \times 4$	 &192  & 106.8  \\
  \hline
\end{tabular}\caption{Comparison of runtime for different block sizes}
\label{table:blocksizes}
\end{table}

It can be seen that for blocks with 192 threads performance is approximately constant across the different block arrangements. This is due to the fact that $BX$ has been chosen to be a multiple of 16 to maximize performance, the parameter $BY$ has much less of an effect on performance and should always be set secondary to $BX$. 

The \textit{CUDA Occupancy Calculator} allows the computation of the multiprocessor occupancy that a particular CUDA kernel has on a particular GPU. The resource usage of the 3D shared memory kernel is shown below in Table \ref{table:occupancy}. Some of these values were attained by compiling with the \texttt{-cubin} option to \texttt{nvcc} (NVIDIA CUDA compiler).

\begin{table}[h]
\centering
\begin{tabular}{ | l | c | }
	\hline
	Resource  											 & Usage \\ \hline
  Threads Per Block	 								& 192 \\
  Registers Per Thread 						 	& 39 \\
  Shared Memory Per Block (bytes)  & 5760 \\
  \hline
\end{tabular}\caption{3D shared memory kernel resource usage - CUDA Occupancy Calculator inputs}
\label{table:occupancy}
\end{table}

The results from the CUDA Occupancy Calculator are shown below in Figure \ref{fig:occupancy}. The kernel uses a large number of registers due to its complexity, and this is what is limiting the maximum number of threads per block to 192. Having said that, adjusting the kernel to use fewer registers, and thus attain higher occupancy, may not necessarily yield higher performance due to the potential introduction of other effects such as additional instructions, spills to device memory and divergent branches.

\begin{figure}
	\centering
		\includegraphics[scale=0.5]{images/occupancy.pdf}
	\caption{The effect of varying the number of threads per block on multiprocessor occupancy}
	\label{fig:occupancy}
\end{figure}


\section{Discussion and Limitations}

\subsection{Speed}
This algorithm does not currently use a narrow band, introduced in Section \ref{upwinding}, to update the level set in either the sequential or parallel versions. If it were introduced into the CUDA kernel, the complexity of the kernel would increase further, increasing the number of registers used. Therefore the performance gains on the parallel algorithm would not be as great in comparison to the gains on the sequential algorithm.

Comparison of CPU and GPU code was done with algorithms that most closely mirrored each other. Although this standardizes the code, it does distort the results slightly as there is potential for optimization on the CPU by making effective use of the CPU cache, and multiple cores (if present). 

Goodman \cite{goodman} shows that CPU code may be slowed if a GPU kernel is executed and therefore suggests that CPU and GPU code run in separate independent environments. To this effect, this has been catered for, increasing the accuracy of the speed up figures attained.

Furthermore, making comparisons with MATLAB code is inadvisable given the environment that MATLAB code runs. MATLAB code is JIT compiled, creating many problems when directly comparing compute times between the two versions. Speed ups were for this reason not measured against MATLAB code. In fact, the main purpose of the MATLAB code was to learn about the inner workings of level set segmentation.


\subsection{Accuracy}
Firstly, the nature of segmenting images using thresholding and curvature terms favours segmentations of anatomical objects with a relatively homogeneous gray value range. Therefore, this algorithm performs best when delineating more homogeneous anatomical structures as can be seen with the detailed brain segmentation shown in Figure \ref{fig:brain3d}.
	
Secondly, as discussed in Section \ref{3dvolumesegmentation} the current 3D level set segmentation solver does not integrate a reinitialization algorithm. This is the largest limitation of this algorithm as it may result in instabilities in the level set function if $\nabla\phi$ values get too large. That said, almost no instabilities were found during testing. It should be noted that when implementing a 3D distance transform there is a major trade off between accuracy and speed. 



