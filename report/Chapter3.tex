\chapter{Implementation}

\section{Level Set Algorithm}\label{levelsetalgorithm}

\subsection{Upwinding}
Equation \eqref{eq:levelsetequation}, the level set equation, needs to be discretized for both sequential and parallel computation. This is done using the \textit{up-wind} differencing scheme. The following explanation of \textit{upwinding} is from \cite{osher2003lsm}.

A first order accurate method for time discretization of equation \eqref{eq:levelsetequation}, is given by the forward Euler method, from \cite{osher2003lsm}:

\begin{equation}
\frac{\phi^{t+\Delta t}-\phi^t}{\Delta t} +F^{t}\cdot{\nabla{\phi^{t}}} = 0
\label{eq:euler1}
\end{equation}

where $\phi^{t}$ represents the current values of $\phi$ at time $t$, $F^{t}$ represents the velocity field at time $t$, and  $\nabla{\phi^{t}}$ represents the values of the gradient of $\phi$ at time $t$. When computing the gradient, a great deal of care must be taken with regards to the spatial derivatives of $\phi$. This is best exemplified by considering the expanded form of equation \eqref{eq:euler1}.

\begin{equation}
\frac{\phi^{t+\Delta t}-\phi^t}{\Delta t} +u^{t}\phi_x^t+v^{t}\phi_y^t+w^{t}\phi_z^t = 0
\label{eq:euler2}
\end{equation}

The one dimensional form of equation \eqref{eq:euler2} at a specific grid point $x_i$ is

\begin{equation}
\frac{\phi^{t+\Delta t}-\phi^t}{\Delta t} +u_i^{t}(\phi_x)_i^t = 0
\label{eq:euler3}
\end{equation}

where $(\phi_x)_i$ is the spatial derivative of $\phi$ at $x_i$. The method of characteristics indicates whether to use a forward difference or backwards difference for $\phi$ based on the sign of $u_i$ at the point $x_i$. If $u_i > 0$, the values of $\phi$ are moving from left to right, and therefore backwards difference methods ($D_x^-$) should be used. Conversely, if $u_i<0$, forward difference methods ($D_x^+$) should be used to approximate $\phi_x$. It is this process of choosing which approximation for the spatial derivative of $\phi$ to use based on the sign of $u_i$ that is known as \textit{upwinding}.

Extending this to three dimensions, from \cite{Lefohn04astreaming}, results in the derivatives below required for the level set equation update. 

\begin{eqnarray}
	D_x &=& (u_{i+1,j,k}-u_{i-1,j,k})/2 \nonumber\\
	D_y &=& (u_{i,j+1,k}-u_{i,j-1,k})/2 \nonumber\\
	D_z &=& (u_{i,j,k+1}-u_{i,j,k-1})/2 \nonumber\\
	D_x^+ &=& u_{i+1,j,k}-u_{i,j,k} \nonumber\\
	D_y^+ &=& u_{i,j+1,k}-u_{i,j,k} \nonumber\\
	D_z^+ &=& u_{i,j,k+1}-u_{i,j,k} \nonumber\\
	D_x^- &=& u_{i,j,k}-u_{i-1,j,k} \nonumber\\
	D_y^- &=& u_{i,j,k}-u_{i,j-1,k} \nonumber\\
	D_z^- &=& u_{i,j,k}-u_{i,j,k-1} \nonumber\\
\end{eqnarray}

$\nabla\phi$ is approximated using the upwind scheme.

\begin{eqnarray}
\nabla\phi_{max} = \left[
  \begin{array}{ c }
     \sqrt{max(D_x^+, 0)^2 + max(-D_x^+,0)^2}  \\[2em]
     \sqrt{max(D_y^+, 0)^2 + max(-D_y^+,0)^2}  \\[2em]
     \sqrt{max(D_z^+, 0)^2 + max(-D_z^+,0)^2}  \\[2em]
  \end{array} \right] \\
\nabla\phi_{min} = \left[
  \begin{array}{ c }
     \sqrt{min(D_x^+, 0)^2 + min(-D_x^+,0)^2}  \\[2em]
     \sqrt{min(D_y^+, 0)^2 + min(-D_y^+,0)^2}  \\[2em]
     \sqrt{min(D_z^+, 0)^2 + min(-D_z^+,0)^2}  \\[2em]  
  \end{array} \right] 
\end{eqnarray}

Curvature is computed using the derivatives below. In two dimensions only the first two derivatives are required.

\begin{eqnarray}
	D_x^{+y} &=& (u_{i+1,j+1,k}-u_{i-1,j+1,k})/2 \nonumber\\
	D_x^{-y} &=& (u_{i+1,j-1,k}-u_{i-1,j-1,k})/2 \nonumber\\
	D_x^{+z} &=& (u_{i+1,j,k+1}-u_{i-1,j,k+1})/2 \nonumber\\
	D_x^{-z} &=& (u_{i+1,j,k-1}-u_{i-1,j,k-1})/2 \nonumber\\
	D_y^{+x} &=& (u_{i+1,j+1,k}-u_{i+1,j-1,k})/2 \nonumber\\
	D_y^{-x} &=& (u_{i-1,j+1,k}-u_{i-1,j-1,k})/2 \nonumber\\
	D_y^{+z} &=& (u_{i,j+1,k+1}-u_{i,j-1,k+1})/2 \nonumber\\
	D_y^{-z} &=& (u_{i,j+1,k-1}-u_{i,j-1,k-1})/2 \nonumber\\
	D_z^{+x} &=& (u_{i+1,j,k+1}-u_{i+1,j,k-1})/2 \nonumber\\
	D_z^{-x} &=& (u_{i-1,j,k+1}-u_{i-1,j,k-1})/2 \nonumber\\
	D_z^{+y} &=& (u_{i,j+1,k+1}-u_{i,j+1,k-1})/2 \nonumber\\
	D_z^{-y} &=& (u_{i,j-1,k+1}-u_{i,j-1,k-1})/2 \nonumber\\
\end{eqnarray}

Using the \textit{difference of normals} method from \cite{Lefohn04astreaming}, curvature is computed using the above derivates using the two normals $\textbf{n}^+$ and $\textbf{n}^-$.

\begin{eqnarray}
\textbf{n}^+ = \left[
  \begin{array}{ c }
     \frac{D_x^+}{\sqrt{(D_x^+)^2 + {(\frac{D_y^{+x}+D_y}{2})}^2 +{(\frac{D_z^{+x}+D_z}{2})}^2  }}  \\[2em]
     \frac{D_y^+}{\sqrt{(D_y^+)^2 + {(\frac{D_x^{+y}+D_x}{2})}^2 +{(\frac{D_z^{+y}+D_z}{2})}^2  }}  \\[2em]
     \frac{D_z^+}{\sqrt{(D_z^+)^2 + {(\frac{D_y^{+z}+D_x}{2})}^2 +{(\frac{D_y^{+z}+D_y}{2})}^2  }}  
  \end{array} \right] &&
\textbf{n}^- = \left[
  \begin{array}{ c }
     \frac{D_x^-}{\sqrt{(D_x^-)^2 + {(\frac{D_y^{-x}+D_y}{2})}^2 +{(\frac{D_z^{-x}+D_z}{2})}^2  }}  \\[2em]
     \frac{D_y^-}{\sqrt{(D_y^-)^2 + {(\frac{D_x^{-y}+D_x}{2})}^2 +{(\frac{D_z^{-y}+D_z}{2})}^2  }}  \\[2em]
     \frac{D_z^-}{\sqrt{(D_z^-)^2 + {(\frac{D_y^{-z}+D_x}{2})}^2 +{(\frac{D_y^{-z}+D_y}{2})}^2  }}  
  \end{array} \right] \nonumber
\end{eqnarray}

The two normals are used to compute divergence, allowing for mean curvature to be computed.

\begin{equation}
H = \frac{1}{2}\nabla\cdot\frac{\nabla\phi}{|\nabla\phi|} = \frac{1}{2}((\textbf{n}_x^+ - \textbf{n}_x^-)+(\textbf{n}_y^+ - \textbf{n}_y^-)+(\textbf{n}_z^+ - \textbf{n}_z^-))
\label{eq:curv}
\end{equation}

\section{Sequential Implementation}
	\subsection{Matlab}
	\subsection{C}

\section{Parallel Implemention}
	\subsection{Block and Grid Sizes}
	\subsection{Shared Memory}